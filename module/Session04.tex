\chapter{Menjalankan dan Menguji Program Solana Secara Lokal}

Untuk menjalankan dan menguji program Solana secara lokal, diperlukan runtime Solana yang dapat dijalankan menggunakan perintah \texttt{test-validator} dari Solana CLI. Perintah ini merupakan \textit{binary} yang memulai kluster Solana dengan satu node lengkap pada workstation. Setelah menginstal Solana CLI, konfigurasi dapat dilakukan untuk menghubungkan alat CLI dengan kluster lokal. Berikut langkah-langkah lengkapnya.

\section{Menjalankan Kluster Lokal}

Perintah \texttt{solana-test-validator} digunakan untuk memulai kluster Solana lokal. Perintah ini akan menjalankan node validator pada workstation dengan semua fitur lengkap yang diperlukan untuk pengujian. Contoh perintahnya adalah sebagai berikut:
\begin{lstlisting}[language=bash, caption={Menjalankan Kluster Lokal}]
	## Memulai kluster lokal
	solana-test-validator
\end{lstlisting}

Node validator ini memulai kluster dengan blok awal yang dikenal sebagai \texttt{Genesis hash}, yaitu blok pertama dalam blockchain Solana.

\section{Memverifikasi Konfigurasi CLI}

Setelah menjalankan validator, konfigurasi alat CLI perlu diverifikasi untuk memastikan bahwa alat CLI terhubung dengan benar ke kluster lokal. Perintah berikut digunakan untuk memverifikasi konfigurasi dengan mencocokkan \texttt{Genesis hash} dari kluster lokal:
\begin{lstlisting}[language=bash, caption={Memverifikasi Konfigurasi CLI}]
	## Memverifikasi Genesis hash
	solana genesis-hash
\end{lstlisting}

Jika konfigurasi berhasil, \texttt{Genesis hash} yang ditampilkan oleh validator akan sesuai dengan keluaran perintah \texttt{solana genesis-hash}.

\section{Mengonfigurasi CLI untuk Berinteraksi dengan Kluster Lokal}

Untuk memastikan alat CLI terhubung ke kluster lokal, konfigurasi URL kluster perlu diatur. Perintah berikut digunakan untuk mengatur URL localhost untuk kluster lokal:
\begin{lstlisting}[language=bash, caption={Mengonfigurasi Alat CLI}]
	## Mengonfigurasi CLI untuk localhost
	solana config set --url http://127.0.0.1:8899
\end{lstlisting}

Setelah konfigurasi, status konfigurasi dapat dilihat menggunakan perintah berikut:
\begin{lstlisting}[language=bash, caption={Melihat Konfigurasi Lokal}]
	## Melihat konfigurasi lokal
	solana config get
\end{lstlisting}

Konfigurasi ini memastikan bahwa alat CLI dapat berkomunikasi dengan node validator yang dijalankan pada workstation lokal.

\section{Penjelasan Contoh dan Output}

\begin{itemize}
	\item Menjalankan perintah \texttt{solana-test-validator} akan memulai node validator lokal, dan terminal akan menampilkan log proses kluster yang dimulai.
	\item Perintah \texttt{solana genesis-hash} akan mencetak hash dari blok awal kluster lokal, yang harus sesuai dengan hash yang tercantum di log validator.
	\item Setelah menjalankan \texttt{solana config set --url http://127.0.0.1:8899}, perintah \texttt{solana config get} akan menampilkan konfigurasi yang telah diatur, termasuk URL kluster lokal.
\end{itemize}

Langkah-langkah ini memastikan lingkungan pengembangan lokal siap untuk mengembangkan dan menguji program Solana secara efisien. Kunci privat dan jalur pasangan kunci akan dibahas lebih lanjut pada sesi berikutnya.
