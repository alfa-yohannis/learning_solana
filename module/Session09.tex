\chapter{Menulis Python Client untuk Berinteraksi dengan Program Solana}

Bab ini membahas cara menulis Python client yang mengirimkan permintaan ke program Solana untuk menyimpan data mata kuliah di blockchain. Python client ini menggunakan modul \texttt{solathon} dan \texttt{soldars}, serta memanfaatkan serialisasi data dengan \texttt{borsh-construct}.

\section{Instalasi Modul Python}
Langkah pertama adalah menginstal modul yang diperlukan. Gunakan perintah berikut:

\begin{lstlisting}[language=bash]
	pip install solathon solders borsh-construct
\end{lstlisting}

\section{Langkah-Langkah Penulisan Python Client}

\subsection{Mengimpor Modul dan Membuat Data Instruksi}
Langkah pertama dalam menulis Python client adalah mengimpor modul yang akan digunakan dan membuat data instruksi yang akan dikirim ke program Solana:

\begin{lstlisting}[language=Python]
	from solders.pubkey import Pubkey
	from solders.system_program import SYS_PROGRAM_ID
	from solathon import Client, Keypair, Transaction
	from solathon.core.instructions import AccountMeta, Instruction
	from borsh_construct import CStruct, U8, String
	
	# Definisi format data untuk serialisasi
	InstructionData = CStruct(
	"variant" / U8,  # Jenis instruksi
	"name" / String,
	"degree" / String,
	"institution" / String,
	"start_date" / String
	)
	
	# Data kursus baru
	data = InstructionData.build({
		"variant": 0,  # Menambahkan kursus baru
		"name": "Pemrograman Lanjut",
		"degree": "Sarjana Teknik",
		"institution": "Pradita University",
		"start_date": "2024-01-10"
	})
\end{lstlisting}

Penjelasan:
\begin{itemize}
	\item \texttt{InstructionData}: Struktur data yang didefinisikan menggunakan \texttt{borsh-construct}. Ini menentukan format data yang akan dikirim.
	\item \texttt{build}: Metode ini digunakan untuk membangun data byte dari dictionary Python.
\end{itemize}

\subsection{Membuat Koneksi ke Cluster Lokal}
Langkah selanjutnya adalah menginisialisasi koneksi ke cluster Solana lokal:

\begin{lstlisting}[language=Python]
	# Inisialisasi client ke localhost
	client = Client("http://127.0.0.1:8899")
\end{lstlisting}

\subsection{Mempersiapkan Akun dan Derivasi PDA}
Kita memerlukan akun-alamat seperti Alice, sistem program, dan PDA. PDA dihasilkan berdasarkan \textit{seed} dan ID program.

\begin{lstlisting}[language=Python]
	# Kunci publik ID program
	program_id = Pubkey.from_string("6aoQLhkzjqr4EPiaQJU2wdXx1J6E5qTLkWkqkFyaNiLo")
	
	# Akun Alice
	alice_keypair = Keypair.from_mnemonic("mnemonic Alice")
	alice_pubkey = alice_keypair.pubkey
	
	# Derivasi alamat PDA
	seed = [b"course", alice_pubkey.to_bytes()]
	pda_pubkey, bump = Pubkey.find_program_address(seed, program_id)
\end{lstlisting}

Penjelasan:
\begin{itemize}
	\item \texttt{Pubkey.find\_program\_address}: Metode dari \texttt{soldars} untuk menghasilkan alamat PDA.
	\item \texttt{seed}: Daftar byte yang digunakan untuk derivasi PDA.
\end{itemize}

\subsection{Membuat Instruksi dan Transaksi}
Setelah semua komponen disiapkan, buat instruksi dan transaksi yang akan dikirim ke program Solana:

\begin{lstlisting}[language=Python]
	# Membuat instruksi
	instruction = Instruction(
	program_id=program_id,
	accounts=[
	AccountMeta(pubkey=alice_pubkey, is_signer=True, is_writable=True),
	AccountMeta(pubkey=pda_pubkey, is_signer=False, is_writable=True),
	AccountMeta(pubkey=SYS_PROGRAM_ID, is_signer=False, is_writable=False),
	],
	data=data
	)
	
	# Membuat transaksi
	transaction = Transaction(instructions=[instruction], signers=[alice_keypair])
	
	# Mengirimkan transaksi ke cluster lokal
	response = client.send_transaction(transaction)
	print("Transaction Signature:", response)
\end{lstlisting}

Penjelasan:
\begin{itemize}
	\item \texttt{Instruction}: Menentukan instruksi yang akan dikirim ke program Solana.
	\item \texttt{Transaction}: Menggabungkan instruksi dan menandatanganinya menggunakan akun Alice.
	\item \texttt{send\_transaction}: Mengirimkan transaksi ke cluster Solana.
\end{itemize}

\subsection{Memverifikasi Data di Log}
Log terminal akan menunjukkan pesan sukses beserta data kursus yang telah disimpan. Periksa log cluster untuk melihat status transaksi dan data yang disimpan.

\section{Menggunakan Curl untuk Verifikasi}
Sebagai tambahan, gunakan \texttt{curl} untuk mendapatkan semua akun program:

\begin{lstlisting}[language=bash]
	curl http://127.0.0.1:8899 -X POST -H "Content-Type: application/json" \
	-d '{"jsonrpc":"2.0","id":1,"method":"getProgramAccounts","params":["ProgramID"]}'
\end{lstlisting}

Hasilnya akan mencantumkan alamat PDA yang dibuat dan data yang disimpan di dalamnya.
