

\chapter{Persiapan Lingkungan Kerja untuk Pengembangan}


Untuk mempersiapkan workstation lokal agar siap digunakan selama mengikuti kursus ini, beberapa alat perlu diinstal pada komputer Anda. Alat-alat yang akan digunakan meliputi Rust, Solana CLI, dan Python. Berikut adalah langkah-langkah instalasi, konfigurasi, dan pengujian untuk setiap komponen.

\section{Instalasi dan Pengujian Rust}

Untuk menginstal Rust, ikuti langkah-langkah berikut:
\begin{enumerate}
\item Unduh penginstal \texttt{RustUp} dari situs resmi Rust di \texttt{https://rustup.rs/}.
\item Jalankan perintah berikut untuk menginstal:
\begin{lstlisting}[language=bash]
curl --proto '=https' --tlsv1.2 -sSf https://sh.rustup.rs | sh
\end{lstlisting}

\item Update Rust:
\begin{lstlisting}[language=bash]
	rustup update stable
\end{lstlisting}

\item Setelah instalasi selesai, tambahkan \texttt{cargo} ke \texttt{PATH}:
\begin{lstlisting}[language=bash]
export PATH="$HOME/.cargo/bin:$PATH"
\end{lstlisting}
\item Perbarui Rust dengan perintah:
\begin{lstlisting}[language=bash]
rustup update
\end{lstlisting}
\item Periksa versi Cargo untuk memastikan instalasi berhasil:
\begin{lstlisting}[language=bash]
cargo --version
\end{lstlisting}
\end{enumerate}

\section{Instalasi dan Pengujian Solana CLI}

Untuk menginstal Solana CLI, ikuti langkah-langkah berikut:
\begin{enumerate}
\item Unduh rilis terbaru dari repositori GitHub Solana CLI di \texttt{https://github.com/solana-labs/solana/releases}.
\item Ekstrak file yang diunduh menggunakan perintah berikut:
\begin{lstlisting}[language=bash]
tar -xvf solana-release-????.tar.bz2
\end{lstlisting}
\item Tambahkan Solana CLI ke \texttt{PATH} dengan menambahkan baris berikut ke skrip startup \texttt{BASH}:
\begin{lstlisting}[language=bash]
export PATH="/path/to/solana/bin:$PATH"
\end{lstlisting}
\item Periksa versi Solana CLI untuk memastikan instalasi berhasil:
\begin{lstlisting}[language=bash]
solana --version
\end{lstlisting}
\end{enumerate}



\section{Instalasi dan Pengujian Python dengan Lingkungan Virtual}

Untuk menginstal dan mengkonfigurasi Python, ikuti langkah-langkah berikut:
\begin{enumerate}
\item Pastikan Python telah terinstal pada sistem Anda. Periksa versi Python dengan perintah:
\begin{lstlisting}[language=bash]
python3 --version
\end{lstlisting}
\item Buat lingkungan virtual baru:
\begin{lstlisting}[language=bash]
python3 -m venv myenv
\end{lstlisting}
\item Aktifkan lingkungan virtual:
\begin{lstlisting}[language=bash]
source myenv/bin/activate
\end{lstlisting}
\item Tingkatkan versi \texttt{pip}:
\begin{lstlisting}[language=bash]
pip install --upgrade pip
\end{lstlisting}
\item Instal modul yang diperlukan, seperti \texttt{Django}, \texttt{Django REST Framework}, dan modul Solana:
\begin{lstlisting}[language=bash]
pip install django djangorestframework solana
\end{lstlisting}
\item Periksa instalasi modul dengan memeriksa versinya:
\begin{lstlisting}[language=bash]
django-admin --version
\end{lstlisting}
\end{enumerate}

Dengan langkah-langkah di atas, semua komponen yang diperlukan telah terinstal dan siap digunakan untuk pengembangan selama kursus ini.




